\documentclass[12pt,a4paper]{article}
\usepackage{geometry}
\geometry{left=2.5cm,right=2.5cm,top=2.0cm,bottom=2.5cm}
\usepackage[english]{babel}
\usepackage{amsmath,amsthm}
\usepackage{amsfonts}
\usepackage[longend,ruled,linesnumbered]{algorithm2e}
\usepackage{fancyhdr}
\usepackage{ctex}
\usepackage{array}
\usepackage{listings}
\usepackage{color}
\usepackage{graphicx}
\usepackage[hidelinks]{hyperref}
% \usepackage[colorlinks=true, linkcolor=blue, urlcolor=blue]{hyperref}  % 自訂連結顏色並移除方框
% Add the following lines
\usepackage{fontspec}
\setmainfont{Sarasa Gothic TC}
\setCJKmainfont{Sarasa Gothic TC}

\begin{document}


\title{
  {
    \heiti 訊號與系統 第 2 次 Homework
  }
}


\date{2024/10/20}
\author{
  年級:{資工三}~~~~~~
  學號:{S11159005}~~~~~~
  姓名:{黃毓峰}~~~~~~
}

\maketitle
\newlength{\question}
\settowidth{\question}{XX}

\section*{\heiti \color{black}{Question - 1.12}}
\noindent {\bf Description}

\parbox[t]{\dimexpr\linewidth-\question}{
  Consider the discrete-time signal
  \[
    x[n] = 1 - \sum_{k=3}^{\infty} \delta[n - 1 - k].
  \]
      Determine the values of the integers \(M\) and \(n_0\) so that \(x[n]\) may be expressed as
  \[
  x[n] = u[Mn - n_0].
  \]
  
}

\noindent {\bf Answer}

\parbox[t]{\dimexpr\linewidth-\question} {
  先來看 \(x[0]\)
  \[
    x[0] = 1 - \sum_{k=3}^{\infty} \delta[0 - 1 - k]
  \]
  for all \(k \geq 3\), \(0 - 1 - k < 0\), so \(\delta[0 - 1 - k] = 0\),
  \(
    x[0] = 1 - 0 = 1.
  \)

  所以重點在於 \(\delta[n - 1 - k]\) 這一個term。

  let \(t \geq 4\),  \(\sum_{k=3}^{\infty} \delta[t - 1 - k] = 1\),  \(x[t] = 0\)

  所以可以得出
  \[
    x[n] = 
      \begin{cases}
          1,              & \text{if } n\geq 4\\
          0,              & \text{otherwise}
      \end{cases}
  \]

  所以 \(x[n]=1- u[n-4]=u[Mn-n_0]\)\\
  經過嘗試\footnotemark{} \(M = -1\), \(n_0 = -3\) 時會成立。
}
\footnotetext{猜數值代進去, 不知道有沒有比較好的方法}

\newpage

\section*{\heiti \color{black}{Question - 1.15}}
\noindent {\bf Description}

\parbox[t]{\dimexpr\linewidth-\question}{
  Consider a system S with input x[n] and output y[n]. This system is obtained through a series interconnection of a system S\(_1\) followed by a system S\(_2\). The input-output relationships for S\(_1\) and S\(_2\) are

  S\(_1\) : \(y_1[n] = 2x_1[n] + 4x_1[n - 1]\),
  
  S\(_2\) : \(y_2[n] = x_2[n - 2] + \frac{1}{2}x_2[n - 3]\),
  
  where x\(_1\)[n] and x\(_2\)[n] denote input signals.
  \noindent \begin{itemize}
    \item[(a)] Determine the input-output relationship for system S.
    \item[(b)] Does the input-output relationship of system S change if the order in which S\(_1\) and S\(_2\) are connected in series is reversed (i.e., if S\(_2\) follows S\(_1\))?
  \end{itemize}
}

\noindent {\bf Answer}

{\bf (a)} 
\parbox[t]{\dimexpr\linewidth-\question} {
由於系統 S 是由系統 S\(_1\) 串聯系統 S\(_2\) 所組成的。
這表示 S\(_2\) 的輸入信號 x\(_2\)[n]  
與系統 S\(_1\) 的輸出信號 y\(_1\)[n] 是相同的, y\(_1\)[n] = x\(_2\)[n]。
\\因此 y[n] 會是:

\[
y[n] = y_2[n] = x_2[n - 2] + \frac{1}{2}x_2[n - 3]
\]

將 x\(_2\)[n] 替換為 y\(_1\)[n],我們得到:

\[
y[n] = y_1[n - 2] + \frac{1}{2}y_1[n - 3]
\]

進一步展開 y\(_1\)[n],我們得到:

\[
y[n] = (2x[n - 2] + 4x[n - 3]) + \frac{1}{2}(2x[n - 3] + 4x[n - 4])
\]

簡化後,我們得到系統 S 的最終input-output relationship:

\[
y[n] = 2x[n - 2] + 5x[n - 3] + 2x[n - 4]
\]
}

{\bf (b)} 
\parbox[t]{\dimexpr\linewidth-\question} {
改變 S\(_1\) 和 S\(_2\) 的連接順序,即讓 S\(_2\) 後接 S\(_1\)。
表示 S\(_2\) 的輸出信號 y\(_2\)[n] 
與系統 S\(_1\) 的輸入信號 x\(_1\)[n] 相同,即 y\(_2\)[n] = x\(_1\)[n]。

因此,我們可以寫出:

\[
y_1[n] = 2x_1[n] + 4x_1[n - 1] = 2y_2[n] + 4y_2[n - 1]
\]

將 y\(_2\)[n] 代入,我們得到:

\[
y[n] = 2(x[n - 2] + \frac{1}{2}x[n - 3]) + 4(x[n - 3] + \frac{1}{2}x[n - 4])
\]

簡化後,我們得到新的系統 S 的input-output relationship:

\[
y[n] = 2x[n - 2] + 5x[n - 3] + 2x[n - 4]
\]

與 (a) 相同。因此系統 S 的input-output relationship不會因為 S\(_1\) 和 S\(_2\) 的連接順序改變而改變。
}



\section*{\heiti \color{black}{SourceCode}}
\sloppy
\noindent \url{https://github.com/IDK-Silver/NUTN-CSIE-Code/blob/main/SignalsAndSystems/hw2/question_and_answer.tex}





\end{document}
