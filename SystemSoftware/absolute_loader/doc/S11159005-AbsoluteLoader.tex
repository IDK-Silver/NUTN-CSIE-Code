\documentclass[UTF8,12pt]{article} % 12pt 為字號大小
\usepackage{amssymb,amsfonts,amsmath,amsthm}
\usepackage{times}
\usepackage{graphicx} % 插圖
\usepackage{cite}
\usepackage{xeCJK}
\usepackage{placeins} % 防止浮動
%----------
% 插入代碼的格式定義
% 參考 https://www.latexstudio.net/archives/5900.html
%----------
\usepackage{listings}
\lstset{
	columns=fixed,
	numbers=left,                                        % 在左側顯示行號
	numberstyle=\tiny\color{gray},                       % 設定行號格式
	frame=none,                                          % 不顯示背景邊框
	backgroundcolor=\color[RGB]{245,245,244},            % 設定背景顏色
	keywordstyle=\color[RGB]{40,40,255},                 % 設定關鍵字顏色
	numberstyle=\footnotesize\color{darkgray},
	commentstyle=\it\color[RGB]{0,96,96},                % 設置代碼註釋的格式
	stringstyle=\rmfamily\slshape\color[RGB]{128,0,0},   % 設置字串格式
	showstringspaces=false,                              % 不顯示字串中的空格
	language=c++,                                        % 設置語言,
	xleftmargin=4em,  % 設置左邊距
	xrightmargin=0em, % 設置右邊距
	tabsize=2,
}
%----------
% 演算法偽代碼
% https://blog.csdn.net/lwb102063/article/details/53046265
%----------
\usepackage{algorithm}
\usepackage{algpseudocode}
\usepackage{amsmath}
\renewcommand{\algorithmicrequire}{\textbf{Input:}}  % Use Input in the format of Algorithm
\renewcommand{\algorithmicensure}{\textbf{Output:}} % Use Output in the format of Algorithm

%----------
% 字體定義
%----------
\setCJKmainfont[BoldFont={Noto Sans CJK TC Bold}, AutoFakeSlant=0.2]{Noto Serif CJK TC}
\setCJKsansfont{Noto Sans CJK TC}
\setCJKfamilyfont{zhsong}{Noto Serif CJK TC}      % 正黑體
\setCJKfamilyfont{zhhei}{Noto Sans CJK TC Bold}   % 粗黑體
\setCJKfamilyfont{zhkai}{Noto Serif CJK TC}       % 斜體楷書(模擬斜體)
\setCJKfamilyfont{zhfs}{Noto Serif CJK TC}        % 仿宋體
\setCJKfamilyfont{zhli}{Noto Serif CJK TC}        % 隸書體
\setCJKfamilyfont{zhyou}{Noto Sans CJK TC}        % 圓體
\newcommand*{\songti}{\CJKfamily{zhsong}} % 正黑體
\newcommand*{\heiti}{\CJKfamily{zhhei}}   % 粗黑體
\newcommand*{\kaiti}{\CJKfamily{zhkai}}   % 斜體楷書(模擬斜體)
\newcommand*{\fangsong}{\CJKfamily{zhfs}} % 仿宋體
\newcommand*{\lishu}{\CJKfamily{zhli}}    % 隸書體
\newcommand*{\yuanti}{\CJKfamily{zhyou}}  % 圓體


%----------
% 版面設置
%----------
%首段縮進
\usepackage{indentfirst}
\setlength{\parindent}{2em}
%行距
\renewcommand{\baselinestretch}{1.25} % 1.25倍行距
%頁邊距
\usepackage[a4paper]{geometry}
\geometry{verbose,
	tmargin=2cm,% 上邊距
	bmargin=2cm,% 下邊距
	lmargin=1cm,% 左邊距
	rmargin=1cm % 右邊距
}

% ----------
% 多級標題格式在此設置
% https://zhuanlan.zhihu.com/p/32712209
% \titleformat{command}[shape]%定義標題類型和標題樣式,字體
% {format}%定義標題格式:字號(大小),加粗,斜體
% {label}%定義標題的標籤,即標題的標號等
% {sep}%定義標題和標號之間的水平距離
% {before-code}%定義標題前的內容
% [after-code]%定義標題後的內容
% ----------
\usepackage{titlesec} %自定義多級標題格式的宏包
% 三級標題
% 4
\titleformat{\section}[block]{\large \bfseries}{\arabic{section}}{1em}{}[]
% 4.1
\titleformat{\subsection}[block]{\normalsize \bfseries}{\arabic{section}.\arabic{subsection}}{1em}{}[]
% 4.1.1
\titleformat{\subsubsection}[block]{\small \mdseries}{\arabic{section}.\arabic{subsection}.\arabic{subsubsection}}{1em}{}[]
\titleformat{\paragraph}[block]{\footnotesize \bfseries}{[\arabic{paragraph}]}{1em}{}[]


%----------
% 其他宏包
%----------
%圖形相關
\usepackage[x11names]{xcolor} % must before tikz, x11names defines RoyalBlue3
\usepackage{graphicx}
\usepackage{pstricks,pst-plot,pst-eps}
\usepackage{subfig}
\def\pgfsysdriver{pgfsys-dvipdfmx.def} % put before tikz
\usepackage[latin1]{inputenc}
\usepackage{tikz}
\usetikzlibrary{shapes,arrows}

%原文照排
\usepackage{verbatim}

%連結的格式
\usepackage[colorlinks,linkcolor=red]{hyperref}
%表格
\usepackage{tabularx}

%==========
% 正文部分
%==========


\begin{document}



\title{\bf{\kaiti 絕對載入器開發報告}}
\author{姓名:黃毓峰\hspace{1cm}學號:S11159005}
\date{2024/06/10}
\maketitle

% \abstract{ }
% \paragraph{\bf{ \kaiti KeyWord}}
% \paragraph{\\}

\section{摘要}
% \textbf{標題大小、樣式可設置,請參閱.tex原始檔\%--標題設定--部分。}

  根據課本的Absolute Loader的設計構想,並利用C來實做出來,該設計出的程式可以讀取以由 ACSII Code 編碼組成的一串16進位數值,並根據給予的記憶體起始位置輸出載入後的記憶體位置,與其改位置的記記憶體數值,可以選在在Shell or Console中顯示或是把結果寫入在文件。

% \subsection{二級標題}
% \subsubsection{三級標題}


\section{流程圖}
 % Define block styles
 \tikzstyle{decision} = [diamond, draw, fill=blue!20,
 text width=4.5em, text badly centered, node distance=3cm, inner sep=0pt]
 \tikzstyle{block} = [rectangle, draw, fill=blue!20,
 text width=5em, text centered, rounded corners, minimum height=4em]
 \tikzstyle{line} = [draw, -latex']
 \tikzstyle{cloud} = [draw, ellipse,fill=red!20, node distance=3cm,
 minimum height=2em]

 \begin{tikzpicture}[node distance = 2cm, auto]
 	% Place nodes
 	\node [block] (parse-argument) {Parse Argument};
 	\node [cloud, left of=parse-argument, node distance=5cm] (start) {Program Start};
 	\node [block, below of=parse-argument] (conversion-hex) {Conversion ACSII to Hex};



 	% \node [block, below of=conversion-hex] (evaluate) {evaluate candidate models};


 	\node [decision, below of=conversion-hex] (decide) {Is write to file};

 	\node [block, left of=decide, node distance=5cm] (open-result-file) {Open result file};

 	\node [block, below of=open-result-file, node distance=3cm] (write-to-file) {Write info to file};
 	\path [line] (open-result-file) -- (write-to-file);
 	\node [block, below of=decide, node distance=3cm] (print-shell) {Print in shell};

 	\node [block, below of=print-shell, node distance=2cm] (end) {End of Program};

 	% Draw edges
 	\path [line] (parse-argument) -- (conversion-hex);
 	\path [line] (conversion-hex) -- (decide);
 	\path [line] (decide) -- node [near start] {yes} (open-result-file);
 	\path [line] (decide) -- node {no}(print-shell);
 	\path [line,dashed] (start) -- (parse-argument);

 	\path [line] (write-to-file) -- (end);
 	\path [line] (print-shell) -- (end);

 \end{tikzpicture}




\section{設計方法}
\subsection{Parse Argument}
定義了一enumeration來標示該程式共有記個argument可以做使用
\begin{lstlisting}[language={C}]
enum ARG_OPTS {
	START_ADDRESS,
	IS_OUT_RESULT,
};
\end{lstlisting}

並定義了一function來把enumeration轉成對應的字串
\begin{lstlisting}[language={C}]
int argtoc(enum ARG_OPTS options, char **result) {
if (result == NULL || *result == NULL) {
	fprintf(
		stderr,
		"argument to string failed : null pointer\n"
	);
	return -1;
}

switch (options) {

	case START_ADDRESS:
	strcpy(*result, "-A");
	return 0;
	return -1;
}
}
\end{lstlisting}

用來存解析結果
\begin{lstlisting}[language={C}]
struct args {
	char* program_path;
	char* start_address;
	char* file_path;
	bool is_out_result;
};
\end{lstlisting}
\newpage
用來則解析程式的argument 的部份程式

\begin{lstlisting}[language={C}]
argtoc(START_ADDRESS, &opt_str);
opt_len = strlen(opt_str);

if (strncmp(arg, opt_str, opt_len) == 0) {

	if (arg_len == opt_len) {
		// move to next argument to get the file path
		arg_index++;

		if (get_arg(argc, argv, arg_index, &arg, &arg_len) < 0) {
			exit(-1);
		}

		result->start_address = arg;
		continue;
	}

	result->start_address = (arg + opt_len);
	continue;
}
\end{lstlisting}
\newpage



\subsection{Conversion ACSII to Hex}
\subsubsection{Number from ASCII}
因不給使用atoi, 以此猜測也不開放使用strtol等等得的function, 故重新設計一把ASCII的字串轉成特定base的function,以下為真對base為16的轉換程式,這樣就可以把Memory Start Address (argument 給的)轉成數字,也可以把輸入文件的ACSII轉成對應的半位元
\begin{lstlisting}[language={C}]
int num_from_ascii(
	const char* accii, const uint32_t ascii_len, const int base
) {
	int result = 0;
	switch (base) {
		case 0x10:
			for (uint32_t index = 0; index < ascii_len; index++) {

				// get the character of string
				char c = accii[index];

				// get dec num from ascii
				c -= '0';

				// more the pure number (9)
				c = (c > (10 - 1)) ? c - (0x10 - (10 - 1)) : c;

				// if character is upper case, sub offset
				c = (c > (0x10 - 1)) ? c - ('a' - 'A') : c;

				// carrying
				result *= base;

				// add number
				result += c;
			}
			break;
		default:
			fprintf(stderr, "num_from_ascii : unknow base");
			return 0;
	}
	return result;
}
\end{lstlisting}


\subsubsection{合併半位元}

為了方便做合併動作定一結構為 hex\_wf (hex with half)

\begin{lstlisting}[language={C}]
struct hex_wf {
	uint8_t rf : 4;
	uint8_t lf : 4;
};
\end{lstlisting}

這樣可以方便且快數(不用做位移以及or bitwise) 可以取得合併2個半位元

\begin{lstlisting}[language={C}]
struct hex_wf* data = (struct hex_wf *)malloc(sizeof(struct hex_wf));
data->lf = 0xf;
data->rf = 0x1;
uint8_t fullbyte = *((uint8_t*) data);
\end{lstlisting}

把合併後的結果存在list裡面並記入list長度, 待全部完成後把list轉成array
\begin{lstlisting}[language={C}]
*dest = (uint8_t*) malloc(sizeof(uint8_t) * (*dest_len));

// to for each element of list
struct list_head *pos, *n;

uint32_t index = 0;
list_for_each_safe(pos, n, &hex_wf_list)
{
	struct list_node_hex_wf *st = list_entry(
		pos,struct list_node_hex_wf, list
	);
	uint8_t content = *((uint8_t*) st->data);
	(*dest)[index] = content;
	index++;
	list_del(pos);
	free(st);
}
\end{lstlisting}


詳細程式碼請見 project\_root/lib/core/loader.c 的 conversion\_hex
\newpage
\subsection{Wirte info to File \& Print in Shell}

藉由上述的步驟可以取的記憶體起始位置,輸入文件載入後的16進位數值,最後要做輸出的部份
為了能夠輸出到不同的IO流又不用寫太多重複的程式碼於是設計了一function可以把資訊寫入到byffer (char array)理面, 最後在輸出到不同的IO流

\begin{lstlisting}[language={C}]
// storage loading file
uint8_t *hex_array  = NULL;
uint32_t hex_len = 0;

// conversion ascii to hex (byte)
conversion_hex(arg_result.file_path, &hex_array,  &hex_len);

// init memory info str buffer
char *buffer = (char *) malloc(sizeof(char) * CONSOLE_MESSAGE_BUFFER_LEN);
memset(buffer, '\0', CONSOLE_MESSAGE_BUFFER_LEN);

// show memory data or wirte to output file
for (int i = 0; i < hex_len; i++) {

	// print info to buffer
	sprint_memory_info(&buffer, address, 5, hex_array[i]);

	// add memory address
	address += sizeof(uint8_t);
	// select IO (stdout, or outputfile)
	if (arg_result.is_out_result) {
		fprintf(out_file, "%s", buffer);
	}
	else {
		fprintf(stdout, "%s", buffer);
	}
}
\end{lstlisting}




\section{開發環境}
\textbf{作業系統 Linux 6.6.32-1-lts x86\_64}

\textbf{桌面環境 KDE Plasma 6.0.5, Qt 6.7.1 , Wayland 1.23.0-1 }

\textbf{CMake 3.29.5 }

\textbf{GCC  14.1.1}

\textbf{Make  4.4.1}

\newpage

\section{討論與心得}
\subsection{討論}
讀文件減數值輸出
\subsection{心得}
最近好累好忙,下次忙的時候不用latex寫學校報告了

\end{document}
